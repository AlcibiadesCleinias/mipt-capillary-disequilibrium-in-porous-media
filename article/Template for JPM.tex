% Please use the review version when submitting papers for review.
% The option below provides the final form of your paper

%\documentclass[review,article,authoryear,jpm]{beg_39}      %  review version
\documentclass[article,authoryear,jpm]{beg_39}             %  final version
% Use option "equation" for numbering equation as section

%\count0=115
\usepackage[hang]{footmisc}
\setlength{\footnotemargin}{0in}
\frenchspacing
\fancypagestyle{plain}{%
  \fancyhf{}
  \fancyhead[R]{\small {\it \jname}, x(x): \thepage--\pageref{LastPage} (\myyear\today)}
  \fancyfoot[R]{\small\bf\thepage }
  \fancyfoot[L]{\fottitle}
  }
%\fancypage{\fbox}{}
\renewcommand{\dmyy}{17}
\renewcommand{\myyear}{2021}
\renewcommand{\today}{}
\begin{document}

\volume{Volume x, Issue x, \myyear\today}
\title{The double porosity model of single-phase flow in a fractured porous medium taking into account scattered fracture of matrix blocks}
\titlehead{The double porosity model with scattered fracture}
\authorhead{O. Izvekov, A. Konyukhov \& I. Cheprasov}
%For at least  authors with different addresses, use instead the following commands
\corrauthor[2]{O. Izvekov}
\author[1]{A. Konyukhov}
\author[2]{I. Cheprasov}
\corremail{izvekov.o@mipt.ru}
\corraddress{Moscow Institute of Physics and Technology, Moscow region, Dolgoprudniy, 141701}
\address[1]{Joint Institute for High Temperatures of the Russian Academy of Sciences, Moscow, Russia, 125412}
\address[2]{Moscow Institute of Physics and Technology, Dolgoprudniy, 141701}

% End information for at least  authors with different addresses
% For authors with the same post address,
%\corrauthor{First A. Author}
%\corremail{f.author@affiliation.com}
%\author{Second B. Author, Jr.}
%\address{Department of Chemistry and Courant, Institute of Mathematical Sciences, New York, NY 10012, USA}
%\corraddress{Department of Chemistry and Courant, Institute of Mathematical Sciences, New York, NY 10012, USA}
% End commands for all authors with the same address

\dataO{mm/dd/yyyy}
%\dataO{}
\dataF{mm/dd/yyyy}
%\dataF{}

\abstract{
The double porosity model of single-phase flow in a deformable medium with an elastic skeleton is generalized to the case of scattered matrix fracture.
The medium consists of low-permeable (in the initial state) blocks and a connected system of mesoscale fractures so that the poroelastic and flow characteristics are rapidly oscillating functions.
The development of small-scale fracturing (scattered fracture) leads to an increase in the permeability of matrix blocks and the intensification of mass transfer between subsystems of double porosity media.
By analyzing the inequality of dissipation, the thermodynamically consistent governing relations and the equation for the evolution of the damage parameter in the matrix are derived.
The resulting system of differential equations includes an equation expressing the condition of equilibrium of the medium, equations for pressures in the subsystems of double porosity, and an equation for the scalar damage parameter, consistent with the dissipation inequality. For the obtained system of equations, an initial-boundary-value problem is formulated and solved numerically that models a coupled processes of flow, fracture, and changes in the stress-strain state in a loaded layer with double porosity, which was initially in equilibrium under high pore pressure.
}

\keywords{Porous medium, Double porosity, Scattered fracture, Abnormally high pore pressure}


\maketitle




\section{Introduction}
\label{intro}
The share of hydrocarbons extracted from unconventional deposits such as shale is increasing from year to year.
The rocks of the fields under consideration are characterized by ultra-low permeability, oil source potential, abnormally high reservoir pressure (overpressure).
The key to the successful production of hydrocarbons from such rocks is the presence of a system of connected fractures in the vicinity of the production well, either technogenic or of natural origin.
The most common development method, originally developed in shale fields in the United States, is horizontal drilling and a series of hydraulic fractures (the so-called multiple fracturing). The ultimate goal is to create a network of secondary fractures in the formation.
The area covered by secondary fractures is called stimulated reservoir volume (SRV) \cite{Warpinski, Wu, Barati, Ma2015}.

Abnormally high reservoir pressure is defined as exceeding the hydrostatic level.
Overpressure is a fairly common phenomenon in formations composed of various rocks, and there are about twenty sources of its origin.
In oil source shale rocks, overpressure is formed mainly due to the generation of fluids (processes of converting kerogen into liquid hydrocarbons).
On the one hand, abnormally high formation pressure can cause catastrophic phenomena, such as the collapse of the borehole walls when drilling with insufficiently dense drilling fluid \cite{Ma2015, Li}.
On the other hand, AHPP can maintain fluid filtration in low permeability conditions when a simple drawdown is not enough for this.
An unusual phenomenon is also associated with abnormal formation pressure.
Wells drilled side by side under seemingly identical conditions can behave differently: one well gushes, showing high flow rates, while the other remains practically dry.
At the same time, the following features are observed: high initial flow rates, sand, and debris production, after the re-start of the well, the flow rates are not restored at the same level \cite{Alekseev}.
All this indicates that in the vicinity of the well there are irreversible destruction processes associated with the presence of abnormal pressure.
Thus, abnormally high formation pressure can be considered as a potentially useful additional source of reservoir energy, the mastering of which is an urgent task.

Continuum damage mechanics (CDM) is a natural approach to describing the nucleation, growth, and coalescence of micro defects, including cracks.
In this case, the number of defects in each elementary volume of the medium should be sufficiently large.
The essence of the approach is the introduction of a scalar or tensor damage parameter into the number of thermodynamic parameters of the state of a continuous medium.
The system of constitutive relations includes a special evolutionary equation for this parameter.
The damage parameter is included in the constitutive relations and affects the behavior of the continuous medium.
The physical interpretation of the damage parameter can be different and depends on specific conditions.
At the macro level, the influence of damage leads to degradation of elastic moduli or the appearance of residual strain.
In the pioneering works of L.M.
Kachanov and Yu.N.
Rabotnov the damage parameter was associated with an effective change in the cross-section of the rod when unloaded areas appeared in the vicinity of microcracks.
Later, the theory of damage acquired a more abstract formulation, bringing it closer to the theory of viscoplasticity \cite{Lemaitre, Murakami}.
In the works of Kondaurov, the role of energy balance in the evolution of micro fracturing is emphasized, and the damage parameter is interpreted as a measure of the dissipation of continuous destruction \cite{Kondaurov2002}.
Subsequently, the model of damage was generalized to the case of saturated porous media with a brittle skeleton \cite{Izvekov} and to the case of a medium with double porosity with a brittle matrix \cite{Izvekov2020}.


The influence of overpressure in a fractured-porous medium on the productivity of wells is usually limited to taking into account the dependence of the permeability of the fracture system on pressure \cite{Thompson}.
CDM had not been applied to this problem until the publication of the paper \cite{Izvekov2020}, although the effect of matrix cracking under the action of overpressure has been known for a long time.
For the first time, the phenomenon of natural hydraulic fracturing (NHF) under the action of overpressure in an impermeable medium was considered in \cite{Secor}.
Later, a generalization was made to the case of a poroelastic medium \cite{Engelder, Luo}.
It is believed that this phenomenon plays an important role in the processes of fluid migration while referring to significant (geological) time scales.
The cyclical nature of the process is noted.
Obviously, a similar effect can be obtained during a technogenic intervention in a reservoir with overpressure, when the initial stress-strain state of the reservoir is disturbed in some way.
In this paper, we consider the case when the overpressure has already formed, but the failure criterion for NHF has not yet been achieved.

The main model for the continual description of fluid flow in a fractured-porous medium is the dual-porosity model, first proposed in the works of Barenblatt, for example, \cite{Barenblatt}.
The family of continuous models of a medium with double porosity is based on the representation of the medium as a superposition of three continua - a solid skeleton and two fluids (fluid in fractures and fluid in matrix).
Mass exchange is possible between fluids (matrix feeds fractures and vice versa).
The skeleton can be poroelastic, poroplastic, or exhibit other rheological properties.
In this work, a continual model of a dual porosity medium is developed.
When the stress state changes under the action of overpressure, microcracks can appear in the matrix, which affects the intensity of mass transfer between the matrix and cracks.

\section{Model definition}
\label{sec:1}
The interest in modeling the movement of fluids in a fractured-porous medium has not weakened over the years, which is due to a wide range of applications.
Cracks are present in geophysical media, artificial materials such as concrete, in biological tissues.
Models differ in the degree of detailing of processes and geometries; cracks are taken into account explicitly or within the framework of the continuum approach.
Continuous models can be divided into single continuum models (effective medium models) and multi-continuum models.
The central issue for effective medium models is the way of upscaling, and the quality of different upscaling approaches for fractured media varies greatly depending on the amount and type of information about the medium and the averaging procedures involved.
In multicontinuous models, a special case of which is the double porosity model, the problem of the mass exchange between subsystems is added \cite{Berre}.


\subsection{General constitutive equations}
\label{sec:2}
Consider a saturated porous medium with an elastic brittle skeleton including a porous matrix and a system of main cracks.
The matrix and fractures are saturated with the same low-compressible fluid.
In the continuum approximation, a dual-porosity medium is a superposition of two fluids and a skeleton.
Fluids correspond to liquids in the matrix and in fractures.
In the present work, the isothermal approximation is used (this assumption is traditional in modeling non-thermal methods of oil production).
All movements are assumed to be quasi-static, and skeletal deformations are small.

Consider a saturated porous medium with an elastic brittle skeleton including a porous matrix and a system of main cracks.
The matrix and fractures are saturated with the same low-compressible fluid.
In the continuum approximation, a dual-porosity medium is a superposition of two fluids and a skeleton.
Fluids correspond to liquids in the matrix and in fractures.
In the present work, the isothermal approximation is used (this assumption is traditional in modeling non-thermal methods of oil production).
All movements are assumed to be quasi-static, and the strain of the skeleton is small.

Differential law of mass conservation in the system under consideration is as follows
\begin{equation}
\frac{\partial {{r}_{1}}}{\partial t}+\nabla \cdot \left( {{r}_{1}}{{\mathbf{v}}_{1}} \right)={{\dot{m}}_{12}},
\end{equation}
\begin{equation}
\frac{\partial {{r}_{2}}}{\partial t}+\nabla \cdot \left( {{r}_{2}}{{\mathbf{v}}_{2}} \right)={{\dot{m}}_{21}},
\end{equation}
\begin{equation}
\frac{\partial {{r}_{S}}}{\partial t}+\nabla \cdot \left( {{r}_{S}}{{\mathbf{v}}_{S}} \right)=0,
\end{equation}
where subscript 1 denotes the fluid in the matrix, subscript 2 denotes the fluid in fractures, subscript S denotes the skeleton, ${{\mathbf{v}}_{A}}$ is velocity of continua, ${{\dot{m}}_{12}}=-{{\dot{m}}_{21}}$ is the rate of mass exchange between the matrix and fractures , ${{r}_{\alpha }}={{\phi }_{\alpha }}{{\rho }_{f}}$ and ${{r}_{S}}=(1-{{\phi }_{1}}-{{\phi }_{2}}){{\rho }_{S}}$, where ${{\rho }_{f}}$ is true density of liquid, ${{\rho }_{S}}$ is density of skeleton material, ${{\phi }_{\alpha }}={dV_{void}^{\alpha }}/{dV}\;$ is the volume fraction of fluids, $dV_{void}^{\alpha }$ is the volume of liquid in subsystem $\alpha=1,2$ and $dV$ is the elementary volume.

Let’s consider the equations of motion of fluids in the form of Darcy’s law.
\begin{equation}
{{\mathbf{W}}_{\alpha }}=-\frac{{{k}_{\alpha }}}{{{\mu }_{f}}}\left( \nabla {{p}_{\alpha }}-{{\rho }_{\alpha }}\mathbf{g} \right),
\end{equation}
where ${{\mathbf{W}}_{\alpha }}={{\phi }_{\alpha }}{{\mathbf{w}}_{\alpha }}={{\phi }_{\alpha }}\left( {{\mathbf{v}}_{\alpha }}-{{\mathbf{v}}_{s}} \right)$, ${{p}_{\alpha }}$ is pore pressure, ${{k}_{\alpha }}$ is permeability, ${{\mu }_{f}}$ is viscosity, $\mathbf{g}$ is gravity acceleration. Let us consider for the simplicity that matrix permeability ${\alpha}=1)$ tends to zero and fracture permeability ${\alpha}=2)$ is constant and scalar.
In general case permeability of fractures depends on stress state and has tensor character.

The equation of equilibrium for the skeleton and the fluids as unified media is as follows
\begin{equation}
\nabla \cdot \mathbf{T}=-r\mathbf{g},
\end{equation}
where $\mathbf{T}={{\mathbf{T}}_{s}}-\left( {{p}_{1}}{{\phi }_{1}}+{{p}_{2}}{{\phi }_{2}} \right)\mathbf{I}$ is the Cauchy total stress, ${{\mathbf{T}}_{s}}$ is the partial stress of skeleton, ${{p}_{\alpha }}$ is pore pressure, the quantity $-{{p}_{\alpha }}{{\phi }_{\alpha }}\mathbf{I}$ is the partial stress of fluids ${{\mathbf{T}}_{\alpha }}$, $r=\sum\limits_{A}{{{r}_{A}}}$, $\mathbf{I}={{\delta }_{ij}}{{\mathbf{e}}_{i}}\otimes {{\mathbf{e}}_{j}}$ is the tensor unit of the second rank, ${{\delta }_{ij}}$ is the Kronecker symbol, tensor multiplication is denoted as $\otimes$.

Clausius-Duhem inequality in the isothermal approximation has the form:
\begin{equation}
-\sum\limits_{A}{{{r}_{A}}\frac{{{d}_{A}}{{\psi }_{A}}}{dt}}+{{\mathbf{T}}_{s}}:\frac{{{d}_{s}}\mathbf{e}}{dt}+\sum\limits_{\alpha }{\left( {{p}_{\alpha }}\frac{{{d}_{s}}{{\phi }_{\alpha }}}{dt}+\frac{{{p}_{\alpha }}{{\phi }_{\alpha }}}{{{\rho }_{\alpha }}}\frac{{{d}_{\alpha }}{{\rho }_{\alpha }}}{dt} \right)}+{{\delta }_{w}}+{{\delta }_{m}}\ge 0,
\end{equation}
where ${{{d}_{A}}\left( ... \right)}/{dt}\;={\partial \left( ... \right)}/{\partial t}\;+{{\mathbf{v}}_{A}}\cdot \nabla (...)$ is the substantial derivative along the path of the particles of the continuum $A=1,2,S$, the quantity ${{\psi }_{A}}$ is Helmholtz’s free energy, $\mathbf{e}=\frac{1}{2}\left( \nabla \otimes \mathbf{u}+\nabla \otimes {{\mathbf{u}}^{T}} \right)$ is the tensor of infinitesimal strain of the skeleton, $\mathbf{u}$ is the displacement vector of skeleton particles, $${{\delta }_{w}}=-\sum\limits_{\alpha }{\mathbf{b}_{\alpha }^{dis}}\cdot {{\mathbf{w}}_{\alpha }}=\sum\limits_{\alpha }{\frac{{{\mu }_{\alpha }}}{{{\phi }_{\alpha }}{{k}_{\alpha }}}{{\left| {{\mathbf{W}}_{\alpha }} \right|}^{2}}}\ge 0$$ is dissipation related to fluid flow, $\mathbf{b}_{\alpha }^{dis}$ is the force of viscous friction between fluids and the skeleton, ${{\delta }_{m}}=-\sum\limits_{\alpha \ne \beta }{{{\chi }_{\alpha }}{{{\dot{m}}}_{\alpha \beta }}}$ is dissipation related to mass exchange, ${{\chi }_{\alpha }}={{\psi }_{\alpha }}+{{{p}_{\alpha }}}/{{{\rho }_{\alpha }}}\;$ specific Gibbs energy, $\mathbf{A}:\mathbf{B}={{A}_{ij}}{{B}_{ij}}$.
When deriving the inequality (6), the law of conservation of energy is taken into account.
In the isothermal case, the energy equation is not used directly and is not presented here.

To close the system of governing laws (1) - (5), we formulate a system of thermodynamically consistent constitutive relations \cite{Truesdell}.
According to the theory developed by Coleman, Noll, and Truesdell, the constitutive relations must satisfy a number of statements including the principle of local action, the principle of invariance from the frame of reference, as well as the principle of thermodynamic consistency.
The latter principle states that the second law of thermodynamics in the form of the Clausius-Duhem inequality (6) imposes a restriction on the possible form of the constitutive relations in an arbitrary process.

As the constitutive relations of a double poroelastic medium with a brittle skeleton, we take the set of the following functions
\begin{equation}
\Upsilon =\Upsilon \left( \mathbf{e},\,{{\rho }_{\alpha }},\,{{\mathbf{w}}_{\alpha }},\,\omega \right),
\end{equation}
where $\Upsilon =\left\{ {{\psi }_{\alpha }},\,{{\mathbf{T}}_{S}},\,{{p}_{\alpha }},\,{{\phi }_{\alpha }},\,{{{\dot{m}}}_{12}},\mathbf{b}_{\alpha }^{dis} \right\}$. Relations (7) should be supplemented with a kinetic equation for the damage parameter $\omega$. We will assume that the accumulation of $\omega$ is irreversible, i.e. healing of microcracks does not occur.
\begin{equation}
\frac{{{d}_{S}}\omega }{dt}=\Omega \left( \mathbf{e},\,{{\rho }_{\alpha }},\,{{\mathbf{w}}_{\alpha }},\,\omega \right),
\end{equation}
where $\Omega$ is the function of the current state of the considered system.

Let’s consider the energy function $$\Psi \left( {{\mathbf{e}}_{s}},\,\,{{p}_{\alpha }},\,\,\omega \right)=F-\sum\limits_{\alpha }{\frac{{{p}_{\alpha }}}{{{r}_{s}}}{{\phi }_{\alpha }}},$$ where $F\left( {{\mathbf{e}}_{s}},\,\,{{\phi }_{\alpha }},\,\,\omega \right)={{\psi }_{s}}\left( {{\mathbf{e}}_{s}},\,\,{{\phi }_{\alpha }}\left( {{\mathbf{e}}_{s}},\,\,{{\rho }_{\alpha }},\,\,\omega \right),\,\,\omega \right)$ Kondaurov, 2007. Necessary and sufficient conditions for the inequality (6) taking into account arbitrary process are as follows:
\begin{equation}
{{\psi }_{\alpha }}={{\psi }_{\alpha }}\left( {{\rho }_{\alpha }} \right); 
\end{equation}
\begin{equation}
{{p}_{\alpha }}=\rho _{\alpha }^{2}\frac{\partial {{\psi }_{\alpha }}}{\partial {{\rho }_{\alpha }}};
\end{equation}
\begin{equation}
\mathbf{T}={{r}_{s}}\frac{\partial \Psi }{\partial \mathbf{e}}; {{\phi }_{\alpha }}=-{{r}_{s}}\frac{\partial \Psi }{\partial {{p}_{\alpha }}}.
\end{equation}
Taking into account relations (9) and (10), inequality (6) takes the form of dissipation inequality
\begin{equation}
{{\delta }_{w}}+{{\delta }_{m}}+{{\delta }_{\omega }}\ge 0,
\end{equation}
where $${{\delta }_{\omega }}=-\left( \frac{\partial \Psi }{\partial \omega } \right)\frac{{{d}_{s}}\omega }{dt}=-\left( \frac{\partial \Psi }{\partial \omega } \right)\Omega$$ is dissipation related to continuum damage.

The procedure for proving relations (9) - (10) is similar to that used in \cite{Truesdell, Kondaurov2007} and is not presented here.
Further, we accept the hypothesis that each dissipation is individually non-negative, which is a sufficient condition for the fulfillment of (11):
\begin{equation}
{{\delta }_{w}}\ge 0, {{\delta }_{m}}\ge 0, {{\delta }_{\omega }}\ge 0.
\end{equation}
This assumption makes it possible to independently impose restrictions on the possible form of the mass exchange law, the fluids flow law, and the evolutionary equation for the damage parameter.
\subsection{Linear approximation}
Let’s consider the special initial state of the system
$\left\{ 0,\,\rho _{\alpha }^{0},\,0,\,\omega \right\}$.
In this state, the skeleton has no strain and pore pressures are zero.
Let’s assume that the pressures in fluids and stress in the skeleton in arbitrary states are small in comparison with the characteristic elastic moduli of the skeleton
${{{p}_{\alpha }}}/{K}\;<<1$, ${\left\| {{\mathbf{T}}_{S}} \right\|}/{K}\;<<1$,
where $K$ is the bulk module of the skeleton. We also consider a special case in which the initial total stresses in the state
$\left\{ 0,\,\rho _{\alpha }^{0},\,0,\,0 \right\}$
are zero. In the case of linear poroelasticity, the assumptions made are correct.

Expansion of the free energy
${{\psi }_{\alpha }}\left( {{\rho }_{\alpha }} \right)$
in a series in a small parameter
${\delta {{\rho }_{\alpha }}}/{\rho _{\alpha }^{0}}\;<<1$
to quadratic terms in the vicinity of the state
$\left\{ 0,\,\rho_{\alpha }^{0},\,0,\,\omega \right\}$
has the next form
\begin{equation}
{{\rho }_{\alpha }}{{\psi }_{\alpha }}\left( {{\rho }_{\alpha }} \right)=\psi _{\alpha }^{0}+p_{\alpha }^{0}\left( \frac{\delta {{\rho }_{\alpha }}}{\rho _{\alpha }^{0}} \right)+\frac{1}{2}{{K}_{f}}{{\left( \frac{\delta {{\rho }_{\alpha }}}{\rho _{\alpha }^{0}} \right)}^{2}},
\end{equation}
where ${{K}_{f}}$ is the bulk module of the liquid. The expression (13) provides linear constitutive relation
\begin{equation}
\frac{\delta {{\rho }_{\alpha }}}{\rho _{\alpha }^{0}}=\frac{\delta {{p}_{\alpha }}}{{{K}_{f}}}.
\end{equation}
It is known that microcracks in shale samples formed during the extraction of hydrocarbons as a result of pyrolysis do not close.
Hence, following \cite{Kondaurov2002, Izvekov, Izvekov2020}, we assume that the main macroscopic effect of damage is the residual strain of the skeleton.
This assumption formally brings this version of the damage model closer to the theory of plasticity.
Let us represent the expansion of the skeleton potential $\Psi $ to values of the second order of smallness, considering the damage to be one of the small parameters, in the following form
\begin{eqnarray}
% \frac{\delta {{\rho }_{\alpha }}}{\rho _{\alpha }^{0}}=\frac{\delta {{p}_{\alpha }}}{{{K}_{f}}},
r_{s}^{0} \Psi (\mathbf{e},{{p}_{\alpha }},\omega )=\frac{1}{2} \lambda I_{1}^{2}+\mu {{J}^{2}}+\sum\limits_{\alpha }{{{b}_{\alpha }}{{I}_{1}}{{p}_{\alpha}}} +\sum\limits_{\alpha }{\frac{1}{2{{N}_{\alpha }}}p_{\alpha }^{2}}+\frac{1}{{{N}_{12}}}{{p}_{1}}{{p}_{2}}+ \nonumber \\ +\sum \limits_{\alpha }{\phi _{\alpha }^{0}{{p}_{\alpha }}}+\gamma \omega +\beta {{{\omega }^{2}}}/{2}\;-\alpha {{I}_{1}}\omega -{{\alpha}_{J}}J \omega -{{\alpha }_{p1}}{{p}_{1}} \omega -{{\alpha }_{p2}}{{p}_{2}} \omega,
\end{eqnarray}
where $\lambda$, $\mu$ are the skeleton Lame coefficients, ${{I}_{1}}=\mathbf{e}:\mathbf{I}$, $J=\sqrt{\mathbf{{e}'}:\mathbf{{e}'}}$, $\mathbf{{e}'}$ is small strain tensor deviator, ${{b}_{\alpha }}$ and ${{N}_{\alpha }}$ are the Biot coefficient and the Biot module, ${{N}_{12}}$ , $\alpha \ge 0$, ${{\alpha }_{J}}\ge 0$, ${{\alpha }_{p1}}\ge 0$, ${{\alpha }_{p2}}\le 0$, $\gamma \ge 0$, $\beta \ge 0$ are phenomenological constants.
The first six terms of (15) represent stored elastic energy of the system, the seven and the eight terms represent work required for new surfaces to appear and the last terms mean released energy when new surfaces appear.

The expression (15) provides linear constitutive relations for the skeleton
\begin{equation}
\mathbf{T}=\left( K{{I}_{1}}-\sum\limits_{\alpha }{{{b}_{\alpha }}{{p}_{\alpha }}-\alpha \omega } \right)\mathbf{I}+\left( 2\mu -\frac{{{\alpha }_{J}}\omega }{J} \right)\mathbf{{e}'},
\end{equation}
\begin{equation}
{{\phi }_{1}}=\phi _{1}^{0}+\left( {{b}_{1}}-\phi _{1}^{0} \right){{I}_{1}}+\frac{{{p}_{1}}}{{{N}_{1}}}+\frac{{{p}_{2}}}{{{N}_{12}}}+{{\alpha }_{p1}}\omega ,
\end{equation}
\begin{equation}
{{\phi }_{2}}=\phi _{2}^{0}+\left( {{b}_{2}}-\phi _{2}^{0} \right){{I}_{1}}+\frac{{{p}_{2}}}{{{N}_{2}}}+\frac{{{p}_{1}}}{{{N}_{12}}}+{{\alpha }_{p2}}\omega ,
\end{equation}
where $K=\lambda +\frac{2}{3}\mu$ is skeleton bulk modulus.

Let us satisfy the inequality (12) using a simple equation that defines the law of mass exchange between fluids
\begin{equation}
\dot{m}=\frac{\rho _{{}}^{0}}{{{\mu }_{f}}}A\left( \omega \right)\left( {{p}_{1}}-{{p}_{2}} \right),
\end{equation}
where $A\left( \omega \right)\ge 0$ is coefficient that determines the intensity of mass transfer.
As a result of cracking, the possibility of liquid transport inside the matrix improves and so mass transfer between fluids increases. The case of an intact matrix (19) coincides with the classical expression \cite{Barenblatt}.

Similarly to \cite{Kondaurov2002, Izvekov, Izvekov2020}, we will accept the following law of damage evolution, which is sufficient to satisfy the dissipation inequality ${{\delta }_{\omega }}\ge 0$
\begin{equation}
\frac{\partial \omega }{\partial t}=-\frac{{{r}_{S}}}{\tau \beta }\left\langle \frac{\partial \Psi }{\partial \omega }
 \right\rangle ,%\,\,\,\left\langle x \right\rangle =\left\{ \begin{align}
%  & x,\,x\ge 0, \\
% & 0,\,x<0. \\
%\end{align} \right.
\end{equation}
where $\left\langle f \right\rangle =0.5(f+|f|)$; $\tau$ is characteristic time.

The physical meaning of the expression (20) is that damage develops only when the elastic energy released when new surfaces appear exceeds the energy consumption for their appearance. The stronger this difference, the more intensively the damage develops. Equation (20) is somewhat analogous to the Griffiths theory for an isolated crack.

Expressions (15) and (20) provide linear evolution law
\begin{equation}
\frac{\partial \omega }{\partial t}=\frac{1}{\tau \beta }\left\langle \alpha {{I}_{1}}+{{\alpha }_{J}}J+{{\alpha }_{p1}}{{p}_{1}}+{{\alpha }_{p2}}{{p}_{2}}-\beta \omega -\gamma \right\rangle.
\end{equation}
Rate of damage accumulation ${\partial \omega }/{\partial t}\;$ should be limited when $\tau \to 0$.
It can be shown that in this case parameter $\omega$ becomes a function of the current state
\begin{equation}
%\omega =\underset{\left[ 0,\,t \right]}
\omega(x) ={\mathop{\max}}\,\left( \frac{1}{\beta }\left( \alpha {{I}_{1}}+{{\alpha }_{J}}J+{{\alpha}_{p1}}{{p}_{1}(x,\tau)}+{{\alpha}_{p2}}{{p}_{2}(x,\tau)}-\gamma  \right) \right),
%\omega =\underset{\left[ 0,\,t \right]}{\mathop{\max }}\,\left( \frac{1}{\beta }\left( \alpha {{I}_{1}}+{{\alpha }_{J}}J+{{\alpha }_{p1}}{{p}_{1}}+{{\alpha }_{p2}}{{p}_{2}}-\gamma \right) \right).
\end{equation}
where $\tau \in [0,t]$.
Simultaneous fulfillment of conditions ${\partial \omega }/{\partial t}\;\ge 0$ and $\omega =0$ gives the following equation for the boundary of the zone of elastic behavior in the space of invariants of the tensor of small deformation
\begin{equation}
\alpha {{I}_{1}}+{{\alpha }_{J}}J+{{\alpha }_{p1}}{{p}_{1}}+{{\alpha }_{p2}}{{p}_{2}}-\gamma =0.
\end{equation}
The system behaves as elastic when $\alpha {{I}_{1}}+{{\alpha }_{J}}J+{{\alpha }_{p1}}{{p}_{1}}+{{\alpha }_{p2}}{{p}_{2}}-\gamma <0$. Active process $({\partial \omega }/{\partial t}\;>0)$ is going on when $\alpha {{I}_{1}}+{{\alpha }_{J}}J+{{\alpha }_{p1}}{{p}_{1}}+{{\alpha }_{p2}}{{p}_{2}}-\gamma >0$. In a passive process (unloading), it is assumed $\omega =const$.

We assume the following restrictions on the model parameters
\begin{equation}
\frac{{{\alpha }^{2}}}{\beta }<K,
\end{equation}
\begin{equation}
{{\left( \frac{1}{{{N}_{12}}}+\frac{{{b}_{1}}{{b}_{2}}}{K} \right)}^{2}}<\left( \frac{1}{{{N}_{1}}}+\frac{\phi _{1}^{0}}{{{K}_{F}}}+\frac{b_{1}^{2}}{K} \right)\left( \frac{1}{{{N}_{2}}}+\frac{\phi _{2}^{0}}{{{K}_{F}}}+\frac{b_{2}^{2}}{K} \right).
\end{equation}
These conditions guarantee the stability of a homogeneous stress state (Izvekov et al, 2020).
Substitution (16) – (19) in (1), (2) and (5) gives
\begin{equation}
\frac{1}{{{M}_{1}}}\frac{\partial {{p}_{1}}}{\partial t}+\frac{1}{{{N}_{12}}}\frac{\partial {{p}_{2}}}{\partial t}+{{b}_{1}}\frac{\partial {{I}_{1}}}{\partial t}+{{\alpha }_{p1}}\frac{\partial \omega }{\partial t}=\frac{A(\omega )}{{{\mu }_{f}}}({{p}_{2}}-{{p}_{1}}),
\end{equation}
\begin{equation}
\frac{1}{{{M}_{2}}}\frac{\partial {{p}_{2}}}{\partial t}+\frac{1}{{{N}_{12}}}\frac{\partial {{p}_{1}}}{\partial t}+{{b}_{2}}\frac{\partial {{I}_{1}}}{\partial t}+{{\alpha }_{p2}}\frac{\partial \omega }{\partial t}-\nabla \left( \frac{{{k}_{2}}({{p}_{2}})}{{{\mu }_{f}}}\nabla {{p}_{2}} \right)=\frac{A(\omega )}{{{\mu }_{f}}}({{p}_{1}}-{{p}_{2}}),
\end{equation}
\begin{equation}
(\lambda +2\mu )\nabla {{I}_{1}}={{b}_{1}}\nabla {{p}_{1}}+{{b}_{2}}\nabla {{p}_{2}}+\alpha \nabla \omega,	
\end{equation}
where $M_{\alpha }^{-1}=N_{\alpha }^{-1}+\phi _{\alpha }^{{}}K_{f}^{-1}$.

\section{Waves of depression and damage}
The system (22), (26) - (28) describes the conjugate processes of fluid flow, deformation, and damage of double porosity saturated the medium.
Of greatest interest are the processes of damage during the propagation of a depression wave from wells or a system of fractures artificially created by means of hydraulic fracturing.
For simplicity, we will consider a one-dimensional process that develops with a decrease in pressure in an infinite plane crack in a porous medium with an initial pore pressure in each of the subsystems $p_0$ and the total stress tensor component in the direction of the normal to the crack $\sigma_0$.
We will neglect the force of gravity.
In this case, the equilibrium equation (28) after integration over the coordinate takes the form
\begin{equation}
{{\sigma }_{0}}=\Lambda {{I}_{1}}-\sum\limits_{\alpha }{{{b}_{\alpha }}{{p}_{\alpha }}-\alpha \omega },
\end{equation}
where $\Lambda =\lambda +2\mu$.
Then, taking into account relation $J=-\sqrt{{2}/{3}\;}{{I}_{1}}$, the damage (22) takes the following form
\begin{equation}
\omega (x,t)={\mathop{\max }}\,\left( {{\chi }_{1}}{{p}_{1}}(x,\tau )+{{\chi }_{2}}{{p}_{2}}(x,\tau )-\Gamma ,\ 0 \right),
\end{equation}
where $\tau \in [0,t]$; $x$ is the coordinate normal to the crack. Following notations are used
\begin{equation}
\Gamma =\frac{\gamma -\left( \alpha -{{\alpha }_{J}}\sqrt{\frac{2}{3}} \right)\frac{{{\sigma }_{0}}}{\Lambda }}{\beta -\left( \alpha -{{\alpha }_{J}}\sqrt{\frac{2}{3}} \right)\frac{\alpha }{\Lambda }},
\end{equation}
\begin{equation}
 {{\chi }_{1}}=\frac{{{\alpha }_{p1}}+\left( \alpha -{{\alpha }_{J}}\sqrt{\frac{2}{3}} \right)\frac{{{b}_{1}}}{\Lambda }}{\beta -\left( \alpha -{{\alpha }_{J}}\sqrt{\frac{2}{3}} \right)\frac{\alpha}{\Lambda }},\quad {{\chi}_{2}}=\frac{{{\alpha }_{p2}}+\left( \alpha -{{\alpha}_{J}}\sqrt{\frac{2}{3}} \right)\frac{{{b}_{2}}}{\Lambda }}{\beta -\left( \alpha -{{\alpha}_{J}}\sqrt{\frac{2}{3}} \right)\frac{\alpha }{\Lambda}}.
\end{equation}
%\begin{center}
\begin{figure}[!b]
	\centering
\includegraphics[width=0.49\columnwidth]{fig2.eps} % \includegraphics[width=12cm]{Fig1.eps}\\
\includegraphics[width=0.49\columnwidth]{fig4.eps} \\
  (a)\hspace*{220pt}(b)
\caption{Damage wave propagation. (a) Case $\chi>0$. (b)  Case $\chi<0$.}
\label{fig1}
\end{figure}
\begin{figure}[!b]
	\centering
\includegraphics[width=0.505\columnwidth]{fig1.eps} % \includegraphics[width=12cm]{Fig1.eps}\\
\includegraphics[width=0.48\columnwidth]{fig6.eps} \\
  (a)\hspace*{220pt}(b)
\caption{ Properties of the flow. (a) Pore pressures $p_1$ and $p_2$ versus distance from the boundary at some point in time. (b) Debit versus time for $\chi=-10$ (solid line) and $\chi=1$ (dashed line).}
\label{fig2}
\end{figure}

According to (30) if mechanical properties of the medium are known, the damage depends only on the history of changes in pore pressures.
We consider case ${{\chi}_{2}}< 0$ since only under this condition damage accumulation is possible.
Neglecting the dependence of the capacitive properties on damage and cross-terms in Eqs. (26), (27) in comparison with the diagonal terms, namely, setting
\begin{equation}
\frac{1}{{{N}_{12}}}+\frac{{{b}_{1}}{{b}_{2}}}{\Lambda}\ll \frac{1}{{{M}_{k}}}+\frac{b_{k}^{2}}{\Lambda},\quad k=1,2
\end{equation}
we arrive at the following system
\begin{equation}
\frac{\partial {{p}_{1}}}{\partial t}=A'(\omega )({{p}_{2}}-{{p}_{1}}),
\end{equation}
\begin{equation}
\frac{\partial {{p}_{2}}}{\partial t}-\frac{\partial }{\partial x}\left( {{K}_{p}}({{p}_{2}})\frac{\partial {{p}_{2}}}{\partial x} \right)=rA'(\omega )({{p}_{1}}-{{p}_{2}}),
\end{equation}
where
\begin{equation}
r=\left( \frac{1}{{{M}_{1}}}+\frac{b_{1}^{2}}{\Lambda} \right)/\left( \frac{1}{{{M}_{2}}}+\frac{b_{2}^{2}}{\Lambda} \right),\quad {{K}_{p}}=\frac{{{k}_{2}}({{p}_{2}})}{{{\mu }_{f}}}/\left( \frac{1}{{{M}_{2}}}+\frac{b_{2}^{2}}{\Lambda} \right),
\end{equation}
\begin{equation}
A'(\omega )=\frac{A(\omega )}{\mu_f}/ \left( \frac{1}{{{M}_{1}}}+\frac{b_{1}^{2}}{\Lambda}\right).
\end{equation}
Let ${{K}_{p}}=const, A'(\omega )={{A}_{\omega }}{{\omega }^{2}}$ for the sake of simplicity.
For a given distance $x$ the two time scales may be defined: characteristic flow time ${{t}_{m}}={{X}^{2}}/{{K}_{p}}$ and time of mass exchange between the subsystems $1/({{A}_{\omega }}\chi_{2}^{2}p_{0}^{2})$.
Equating these times we define the length scale ${{X}_{m}}={{K}_{p}}^{1/2}/({{A}_{\omega }}^{1/2}{|{\chi }_{2}|}{{p}_{0}})$. Using scales ${t}_{m}$, ${X}_{m}$, $p_0$ for pressure and ${|\chi }_{2}|{{p}_{0}}$ for damage the equations can be written in the dimensionless form:
\begin{equation}
\frac{\partial {{p}_{1}}}{\partial t}={{\omega }^{2}}({{p}_{2}}-{{p}_{1}}),
\end{equation}
\begin{equation}
\frac{\partial {{p}_{2}}}{\partial t}-\frac{{{\partial }^{2}}{{p}_{2}}}{\partial {{x}^{2}}}=r{{\omega }^{2}}({{p}_{1}}-{{p}_{2}}),
\end{equation}
If ${{\chi}_{2}}< 0$ we can write
\begin{equation}
\omega (x,t)={\mathop{\max }}\,\left(  \chi {{p}_{1}}(x,\tau )-{{p}_{2}}(x,\tau )-{\Gamma }',\ 0 \right), \tau \in [0,t].
\end{equation}
Here, the dimensionless parameters $\chi ={{\chi }_{1}}/{|{\chi }_{2}|},\ {\Gamma }'=\Gamma /({|{\chi }_{2}|}{{p}_{0}})$ are introduced.

Depending on the sign of $\chi$ two types of solutions are possible.
Case $\chi<0$.
The damage begins to accumulate as a result of a decrease in pressure in subsystem 2.
A decrease in pressure in subsystem 1 accelerates the destruction process.
Since at each point in space the pore pressure decreases, asymptotically approaching the pressure in the crack, the fracture process, in this case, is unlimited in time.
The structure of the damage wave is shown in Fig. 1~a.
Case $\chi>0$.
As in the previous case, damage begins to accumulate as a result of depression in subsystem 2.
A decrease in pressure in subsystem 1 slows down the destruction process. In the process of equalizing the pressures, the growth of damage stops.
The maximum pressure difference in the subsystems and, accordingly, the accumulated damage decreases with the distance to the boundary.
The evolution of damage is shown in Fig. 1~b.
The effect of a scattered fracture on the flow rate when producing from a formation with abnormally high pore pressure and low permeability is of great interest.
Pore pressures $p_1$ and $p_2$ versus distance from the boundary at some point in time are shown in Fig. 2~a.
The debit dependence on time in a wide range of parameter $r$ is shown in Fig. 2~b for $\chi=-10$ (solid line) and $\chi=1$ (dashed line).
Linear dependence with slope $-1/2$ corresponds to solution without damage accumulation ($r=0$). It follows from the figure that all solutions have this asymptotic $q \propto 1/\sqrt{t}$. At the same time, the initial stage at $t < 1$ is characterized by nearly constant ($\chi=-1$) or even increasing in time ($\chi=-10$) flow rate, which depends on the value of $r$.
\textit{Remark.} It is shown in \cite{Izvekov2020} that for shales following estimates hold. Value of ${{{\alpha }_{J}}}/{\alpha }\approx {3-7\nu +2{{\nu }^{2}}}/{\sqrt{6}{{\left( 1-\nu  \right)}^{2}}}$ belongs to the interval $1.07 -0.77$ at $\nu =0.2-0.35$, where $\nu$ is the Poisson coefficient of matrix. The following conditions hold ${{\alpha }_{p2}}\approx -{{\alpha }_{p1}}{{b}_{2}}\le 0$, ${{{b}_{1}}}/{{{b}_{2}}}\;\approx 10$, where ${{b}_{1}}\approx 1$ is close to the Biot coefficient of the matrix. In this case $\chi <0$ is of order $-10$.

\section{Conclusions}
%Extensive literature is devoted to the problem of the influence of the stressed state of the reservoir on the permeability of rocks, in particular shale, [2-9]. A non-linear and even non-monotonic character of the effect of stresses on the permeability of the fracture system is noted. On the one hand, under conditions of all-round compression, formation depletion leads to fracture closure and a decrease in effective permeability. On the other hand, if the stress state is not hydrostatic, a decrease in pressure can lead to shear along the crack faces and an increase in permeability due to the phenomenon of dilatancy [6]. In recent years, much attention has been paid to numerical modeling (using the finite and boundary element method) of regular and irregular 2D and 3D fracture systems in terms of their permeability [5,7,8].




%\label{sec:ProblemDef} In this section, we follow the notation in~\cite{Xiang,Ghanem}.  Define a complete probability space $(\Omega,\mathcal{F},\mathcal{P})$ with sample space $\Omega$ which corresponds to the outcomes of some experiments, $\mathcal{F}\subset 2^\Omega$ is the $\sigma$-algebra of subsets in $\Omega$ (these subsets are called events) and $\mathcal{P}:\mathcal{F}\rightarrow[0,1]$ is the probability measure. Also, define $D$ as a $d$-dimensional bounded domain $D\subset\mathbb{R}^d \ (d=1,2,3)$ with boundary $\partial D$. We are interested to find a stochastic function $u:\Omega \times D \rightarrow \mathbb{R}$ such that for $\mathcal{P}$-almost everywhere (a.e.) $\omega \in \Omega$, the following equation holds:
%\begin{equation}
%  \mathcal{L}(\mbox{\boldmath $x$},\omega;u) = f(\mbox{\boldmath $x$},\omega),~~\forall \mbox{\boldmath $x$} \in D,
%  \label{eqn:SDE}
%\end{equation}
%\noindent
%and
%\begin{equation}
%  \mathcal{B}(\mbox{\boldmath $x$};u) = g(\mbox{\boldmath $x$}),~~\forall \mbox{\boldmath $x$} \in \partial D,
%  \label{eqn:Boundary}
%\end{equation}
%\noindent
%where $\mbox{\boldmath $x$} = (x_1,\dots,x_d)$ are the coordinates
%in $\mathbb{R}^d$, $\mathcal{L}$ is (linear/nonlinear) differential operator, and $\mathcal{B}$ is a boundary operator. In the most general case, the operators $\mathcal{L}$ and $\mathcal{B}$ as well as the driving terms $f$ and $g$, can be assumed random. We assume that the boundary has sufficient regularity and that $f$ and $g$ are properly defined such that the problem in Eqs.~(\ref{eqn:SDE})--(\ref{eqn:Boundary}) is well-posed $\mathcal{P}$ -a.e. $\omega \in \Omega$.




%\subsection{The Finite-Dimensional Noise Assumption and the Karhunen-Lo\`eve Expansion the Finite-Dimensional Noise Assumption and the Karhunen-Lo\`eve Expansion}\label{sec:KLE} Any second-order stochastic process can be represented as a random variable at each spatial and temporal location. Therefore, we require an infinite number of random variables to completely characterize a stochastic process. This poses a numerical challenge in modeling uncertainty in physical quantities that have spatio-temporal variations, hence necessitating the need for a reduced-order representation (i.e., reducing the infinite-dimensional probability space to a finite-dimensional one). Such a procedure, commonly known as a `finite-dimensional noise assumption' \cite{FooPhD,2006AIPC..845..479B}, can be achieved through any truncated spectral expansion of the stochastic process in the probability space. One such choice  is the Karhunen-Lo\`eve (K-L) expansion \cite{Ghanem}. \subsubsection{Suspended Structures} Suspended structures used for nanowire thermal conductivity measurements serve as a good example. From the entire volume, it is clear that significant progress has been made in experimental techniques to probe nanoscale heat transfer phenomena, and the experimental results have led to new understandings of heat transfer physics, generated new challenges, and opened new opportunities. From my own perspective, the following are some significant challenges. \paragraph{Suspended Structures} Suspended structures used for nanowire thermal conductivity measurements serve as a good example. From the entire volume, it is clear that significant progress has been made in experimental techniques to probe nanoscale heat transfer phenomena, and the experimental results have led to new understandings of heat transfer physics, generated new challenges, and opened new opportunities. From my own\footnote{Suspended structures used for nanowire thermal conductivity measurements serve as a good example.} perspective, the following are some significant challenges.

%\begin{theorem}\label{un_energy}
%There exists a unique solution $u_n \in L^2\left(O, H_0^1\left(D\right)\right)$ to the problem \eqref{eqn:SDE} and \eqref{eqn:Boundary} for $n=0$,
%and the problem \eqref{eqn:SDE}--\eqref{un_est} for $n\geq 1$. In addition, if $f\in L^2\left(O, H^{-1+\sigma}\left(D\right)\right)$ for $\sigma\in(0,1]$, it holds that
%\begin{equation}\label{un_est}
%E(|{u_n}|_{H^{1+\sigma}(D)}^2) \leq  C_0^{n+1} \;E({f}_{H^{-1+\sigma}(D)}^2),
%\end{equation}
%for some constant $C_0$ independent of $n$ and $s$.
%\end{theorem}

%\medskip

%\begin{proof}
%For $n=0$, the existence of the weak solutions can be deduced from the Lax-Milgram theorem,
%and the desired energy estimate,
%\begin{equation*}
%E(|{u_0}|_{H^{1+\sigma}(D)}^2) \leq  \tilde C_0\;E({f}_{H^{-1+\sigma}(D)}^2),
%\end{equation*}
%This completes the proof.
%\end{proof}






%% The Acknowledgements part is started with the command \acknowledgements;
%% acknowledgements are then done as normal sections before appendix
%% \acknowledgements

\acknowledgements

This research was supported by the grant in the form of a subsidy for a large scientific project in priority areas of scientific and technological development No. 13.1902.21.0035.


%% The Appendices part is started with the command \appendix;
%% appendix sections are then done as normal sections and after Acknowledgements
%% \appendix

%% \section{}
%% \label{}

%% References without bibTeX database:

\begin{thebibliography}{-8}

\small{
\bibitem{Warpinski}Warpinski N.R., Mayerhofer M.J., Vincent M.C., Cipolla C.L., and Lolon E.P., Stimulating unconventional reservoirs: maximizing network growth while optimizing fracture conductivity, J. Can. Pet. Technol., 2009, vol. 48, no. 10, pp. 39–-51.

\bibitem{Wu}Wu Y.S., Li J., Ding D., Wang C., and Di Y., A generalized framework model for the simulation of gas production in unconventional gas reservoirs, SPE J., 2014, vol. 19, no. 5, pp. 845-–857.

\bibitem{Barati}Barati R. and Liang J.T., A review of fracturing fluid systems used for hydraulic fracturing of oil and gas wells, J. Appl. Polym. Sci., 2014, vol. 131, no. 16, Paper ID 40735.
    
\bibitem{Ma}Ma X. and Zabaras N., An adaptive hierarchical sparse grid collocation algorithm for the solution of stochastic differential equations, Journal of Computational Physics, vol. 228, no. 8, pp. 3084-–3113, 2009.

\bibitem{Ma2015}Ma Y. Zee, and Holditch S. Unconventional oil and gas resources handbook: Evaluation and development. Gulf professional publishing, 2015.

\bibitem{Li}Li S., George J., and Prudy C., Pore-pressure and wellbore stability prediction to increase drilling efficiency, J. Pet. Technol., 2012, V. 64, no. 02, P. 98-–101.

\bibitem{Alekseev}Alekseev A.D. Bazhenov Formation: In Search of Large Shale Oil in Upper Salym, Rogtec Magazine. 2013. V. 34. P. 15--39. (in Russian)

\bibitem{Lemaitre}Lemaitre J. A Course on Damage Mechanics. 228 p. Springer-Verlag Berlin Heidelberg, 1996.

\bibitem{Murakami}Murakami S. Continuum Damage Mechanics. Springer Netherlands. 2012.  402 p.

\bibitem{Kondaurov2002}Kondaurov V.I., Fortov V.E. Fundamentals of Thermomechanics of Condensed Matter. 2002. MIPT. Moscow(in Russian)

\bibitem{Izvekov}Izvekov O. Y. and Kondaurov V. I. (2009). Model of a porous medium with an elastic fractured skeleton. Izvestiya, Physics of the Solid Earth, 45(4), 301--312.

\bibitem{Izvekov2020}Izvekov O. Y., Konyukhov A. V. and Cheprasov I. A. (2020). Thermodynamically Consistent Filtration Model in a Double Porosity Medium with Scattered Fracture of a Matrix. Izvestiya, Physics of the Solid Earth, 56(5), 695--707.
    
\bibitem{Thompson}Thompson J.M., Nobakht M. and Anderson D.M. Modeling well performance data from overpressured shale gas reservoirs // Canadian Unconventional Resources and International Petroleum Conference. Society of Petroleum Engineer. 2010. SPE-137755-MS.

\bibitem{Secor}Secor Jr. D.T. Role of Fuid pressure in jointing. American Journal of Science. 1965. V. 263. no. 8. P. 633-646.

\bibitem{Engelder}Engelder T., Lacazette A. Natural hydraulic fracturing. P. 35–-44. Rock joints: Rotterdam, AA Balkema. 1990.

\bibitem{Luo}Luo X., Vasseur G. Natural hydraulic cracking: numerical model and sensitivity study. Earth and Planetary Science Letters. 2002. V. 201. no. 2. P. 431–-446.

\bibitem{Barenblatt}Barenblatt G.I., Zheltov Y.P., Kochina I.N. 1960. Basic concepts in the theory of seepage of homogeneous liquids in fissured rocks. PMM. V. 24. no. 5. P. 852-—864.

\bibitem{Berre}Berre I., Doster F. and Keilegavlen E. (2019). Flow in fractured porous media: a review of conceptual models and discretization approaches. Transport in Porous Media, 130(1), 215--236.

\bibitem{Truesdell}Truesdell C. and Noll W. The non-linear field theories of mechanics. In The non-linear field theories of mechanics (pp. 1--579). Springer, Berlin, Heidelberg.(2004).

\bibitem{Kondaurov2007}Kondaurov V.I. Mechanics and thermodynamics of a saturated porous medium. Moscow, MIPT. 2007 (in Russian)



}

\end{thebibliography}

%% References with bibTeX database:

%\bibliographystyle{Bibliography_Style}

%\bibliography{References}
\end{document}
